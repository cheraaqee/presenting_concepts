\documentclass{article}
\usepackage{geometry}
\geometry{a4paper, top=2cm, left=2cm, right=2cm, left=2cm}
\usepackage{hyperref}
\usepackage{framed}
\usepackage{xcolor}
\usepackage{xepersian}
\settextfont{XB Niloofar}
\begin{document}
\title{پیشنهاده‌ی طرح صنعتی}
\maketitle{}
%%%%%%%%%%%%%%%%%%%%%%
\begin{framed}
	\section{عنوان طرح}
	\subsection{عنوان طرح به فارسی:}
	\textcolor{gray}{راه‌اندازیِ مرکز داده برای گروه مهندسی کامپیوتر دانشگاه گیلان}
	\subsection{عنوان طرح به انگلیسی:}
	\begin{flushleft}
	\textcolor{gray}{\lr{Setting Up a Data Center for the Computer Engineering Department of the Univeristy of Guilan}}
	\end{flushleft}
\end{framed}
%%%%%%%%%%%%%%%%%%%%%%%%%%%
\begin{framed}
	\section{کارفرما}
	\subsection{مشخصّات کارفرما}
	\textcolor{gray}{اسم، آدرس، تلفن و ...}

	\noindent \textcolor{gray}{مثلاً گروهِ کامپیوتر دانشگاه گیلان}

	\textcolor{gray}{گیلان-رشت- کیلومتر پنج بزرگراه خلیج فارس- مجتمع آموزشی دانشگاه گیلان- دانشکده‌ی فنّی- ... کدپستی...}

	\textcolor{gray}{تلفن: ...۰۱۳}
	
        \textcolor{gray}{وبگاه}
\end{framed}
%%%%%%%%%%%%%%%%%%%%%%%
\begin{framed}
	\section{مجری}
	\subsection{نام حقوقی/حقیقی مجری/مجریان}
	\textcolor{gray}{مثلاً شرکت داده پردازان قوی (در صورت حقوقی بودن)}

	\textcolor{gray}{یا مثلاً اسامیِ مجریان (در صورت حقیقی بودن)}
	\textcolor{gray}{
	\begin{enumerate}
		\item \textcolor{gray}{علی قلی}
		\item \textcolor{gray}{گلی قلی}
		\item \textcolor{gray}{$\ldots$}
	\end{enumerate}}
	\subsection{مشخّصات و سوابق مجری}
	\textcolor{gray}{تحصیلات، طرح‌های انجام‌شده، سابقه‌ی کاری و ...}
	\subsection{اطلاعات تماس}
	\textcolor{gray}{ایمیل، آدرس، شماره تماس، تلفن ثابت، وبگاه (حالا واقعاً نمی‌خواد بنویسین واسه شیوه، ولی جای دیگه احتمالاً می‌خواد. البتّه بعضی جاها خودشون قالب می‌دن که پیشنهاده رُ باید طبقِ همون تکمیل کنین.)}
\end{framed}
%%%%%%%%%%%%%%%%%%%%%%
\begin{framed}
	\section{تعریف مسئله}
	\label{sec:definition}
	\textcolor{gray}{در این قسمت هم یک فرد غیر فنّی باید بفهمه مسئله چیه، هم افرادِ متخصّص. اوّل تعریف کلّی، بعد جزئیّاتِ تخصّصی.}

	\textcolor{gray}{کارکردِ این قسمت اینه که کارفرما و کارشناس‌های فنّیِ مُعتمدِش بفهمن که واقعاً یه مشکلی وجود داره که باید بهش پرداخته بشه. یا اینکه متوجّه بشن یه فنّاوری‌ای وجود داره که می‌تونه کارهاشون رُ تسهیل کنه.}

\end{framed}
%%%%%%%%%%%%%%%%
\begin{framed}
	\section{کارهای انجام شده}
	\label{sec:review}
	\textcolor{gray}{کارکردِ این قسمت اینه که کارفرما و کارشناس‌های فنّیِ معتمدش، بفهمن که اگه شما طرحُ انجام ندین، با چه هزینه‌ای می‌تونن چه محصولی رُ استفاده کنن. }
	\subsection{محصول‌های تجاری}
	\textcolor{gray}{آدرس می‌دین، قیمت می‌گین، نقاط قوّت و ضعف‌شون رُ تحلیل می‌کنین. {\Large توجّه:} توی تحلیل کارهای گذشته، نظر شخصی‌تون رُ نمی‌گین، به آمار و اسناد رجوع می‌کنین. خیلی رُک، می‌گین فلان شرکت، محصولُ با این قیمت ارائه می‌کنه، بعداً شما، توی هزینه‌ها، قیمتِ کمتری پیشنهاد می‌دین. یا مثلاً خدمات پس از فروشِ یه شرکتی، ۲ ساله‌ست، شما بگین، ما بیشتر می‌دیم. البته تحلیل فنّی هم هست دیگه، مثلاً فلان نرم‌افزار از یه کتابخونه استفاده می‌کنه مشکلاتِ امنیتی‌ش اخیراً کشف شدن، ما از اون کتابخونه استفاده نمی‌کنیم؛ همه‌ی کدشُ خودمون می‌زنیم!}
	\subsection{پژوهش‌های دانشگاهی}
	\textcolor{gray}{مقاله‌ها و پایان‌نامه‌ها و ... اینارُ اینجا مرور می‌کنین. کُدهای آماده تو گیت‌هابُ جاهای دیگه رُ هم باید امتحان کرده باشین و تحلیل کنین. (بالاخره زمینه‌ی تخصّصی‌تونه دیگه، باید بدونین توش چه خبره و دنیا دستِ کیه)}
\end{framed}
%%%%%%%%%%%%%%%%%%%%%%%%%%%%%%
\begin{framed}
	\section{روش پیشنهادی}
	\textcolor{gray}{اینجا باید بگین که چجوری می‌خواین اون مسئله‌ای که تو قسمت~\ref{sec:definition} تعریف کردین رُ حل کنین. کارفرما و متخصّص‌هاش باید بفهمن که روش شما معقول و قابل اجراست و وعده‌ی سر خرمن نمی‌دین.}

	\textcolor{gray}{برتری‌های فنّیِ روش شما باید تو این قسمت عیان باشه، ولی به‌طور ضمنی؛ کسی که متخصّصه، خودش باید بفهمه چه خوبی‌هایی داره روشِ شما.}
\end{framed}
%%%%%%%%%%%%%%%%%%%%%%%
%%%%%%%%%%%%%%%%%%%%%%
\begin{framed}
	\section{مراحل و زمان‌بندی اجرای طرح}
	\textcolor{gray}{وقتی یه طرحی واسه کسی انجام می‌دین، نباید بگین: \emph{خب قراردادُ ببند، پولُ بده، من کارُ آخرسر تحویل‌ت می‌دم.} کارفرما داره سرمایه‌گذاری می‌کنه، نگرانِ سرمایه و وقت‌شه؛ دوست داره بدونه شما مراحل انجام کارتون چیه؟ هر مرحله چقدر طول می‌کشه؛ آخر هر مرحله، چه خروجی‌ای باید حاصل شده باشه؟}

	\textcolor{gray}{باید یه همچین تعهّدی تو این قسمت به کارفرما بدین و زمان‌بندی‌تونُ مشخّص کنین. یه \href{https://duckduckgo.com/?t=ffab\&q=what+is+a+gantt+chart\&atb=v342-1\&ia=web}{نمودارِ گانت} هم \href{https://texdoc.org/serve/pgfgantt.pdf/0}{می‌کشین} اینجا.}
\end{framed}
%%%%%%%%%%%%%%%%%%%%%
%%%%%%%%%%%%%%%%%%%%%%
\begin{framed}
	\section{ملزومات و هزینه‌ها}
	\textcolor{gray}{کارکردِ این قسمت اینه که کارفرما بفهمه برای اینکه شما طرحُ براش انجام بدین، چی کارا باید بکنه.}
	\subsection{ملزومات}
	\textcolor{gray}{مثلاً ممکنه بگین من برای جمع‌آوری مجموعه‌داده، نیازدارم همه‌ی کارکنان سازمانِ کارفرما به-خط بشن، تا من ازشون نمونه بگیرم. یا مثلاً بگین برای پر کردن پایگاه‌داده، از همه باید مصاحبه بگیرم.}
	\subsection{هزینه‌ها}
	\textcolor{gray}{هزینه‌های مالی رُ اینجا فهرست کنین. چقدر پول باید بده کارفرما و این پول‌هارُ بابت چیا باید بده. یه جدول درست کنین، مثل این:}
	\vskip 0.5cm
	\begin{center}
	\textcolor{gray}{\scriptsize \begin{tabular}{||r|r|r||}\hline \hline ردیف&مورد&هزینه (دلاری)\\ \hline ۱&دستمزد روزانه‌ی کاربر غیرمتخصّص برای برچسب زدن داده‌ها و ساخت مجموعه‌داده\end{tabular}}
	\end{center}
	\vskip 0.5cm
	\textcolor{gray}{یادتون باشه جدولتون شماره و عنوان داشته باشه.}

	\textcolor{gray}{اگه طرح‌تون تو چند مرحله اجرا میشه، هزینه‌هارُ واسه هر مرحله جدا قید کنین.}

	\textcolor{gray}{در نهایت  هم یه «مجموعِ هزینه‌ها» ارائه کنین؛ یا تو همون جدول یا به‌صورت جدا}

\end{framed}
%%%%%%%%%%%%%%%%%%%%%
%%%%%% REF %%%%%%%%%%
\begin{framed}
	\section{فهرست منابع و مراجع}
	\textcolor{gray}{تو قسمت~\ref{sec:review} به منابع ارجاع می‌دین دیگه؛ همینجوری که اسمِ مقاله رُ نمیارین.}
\end{framed}
%%%%%% REF %%%%%%%%%%%
\end{document}
