\documentclass{article}
\usepackage{hyperref}
\usepackage{xepersian}
\settextfont{IRLotus}
\begin{document}
\title{تصمیم‌گیری در مورد افزایش تعداد اتوبوس‌های ایاب و ذهاب}
\author{فرناز میرنظامی}
\maketitle
\section{مقدمه} \label{sec:intro}
از آن‌جایی که دانشگاه گیلان خارج شهر واقع شده و تعدادی از خوابگاه‌ها خارج از محوطه دانشگاه هستند، نیاز به اتوبوس‌های ایاب و ذهاب، نیازی ضروری است. در حال حاضر تعدادی اتوبوس وجود دارد، امّا با توجه به تعداد زیاد دانشجو و رفت و آمد‌شان در ساعت‌های مختلف نیاز به اتوبوس‌های بیشتری هست.

\section{نظر}
با توجه به توضیحات قسمت \ref{sec:intro}، 
بهتر است برای هر خوابگاه، یک ایستگاه درنظر بگیریم و دو نوع اتوبوس داشته باشیم؛ یکی برای رفتن به شهر و دیگری برای رفتن به دانشگاه. در ضمن برای اینکه بتوان این نیاز را تأمین کرد، بهتر است برآورد میانگینی از تعداد دانشجوها در هر روز داشته باشیم. (مثلاً بر اساس یک پرسشنامه برخط\RTLfootnote{معادل فارسی «on-line»} راجع به ساعت و روز کلاس‌ها) که با توجه به تعداد زیاد دانشجو، به تعداد بیشتری اتوبوس نیاز خواهیم داشت.

با در نظر گرفتن مسافت هر خوابگاه و زمان رسیدن به دانشگاه، ممکن است برای بعضی ایستگاه‌ها به این نتیجه برسیم که به اتوبوس بیشتری نیاز داریم.
\section{جمع‌بندی}
اگر تعداد اتوبوس‌ها زیادتر شود، در ساعت‌های بیشتری رفت و آمد ممکن است و نگرانی دانشجویان بابت عقب افتادن کلاس کمتر می‌شود و همچنین برای خرید و سایر کارها، در هر ساعتی می‌توانند به شهر رفت و آمد کنند.
\end{document}
