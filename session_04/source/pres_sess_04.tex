\documentclass{article}
\usepackage{hyperref}
\usepackage{xcolor}
\usepackage{xepersian}
\settextfont{XB Niloofar}
\begin{document}
\baselineskip=0.8cm
\textwidthfootnoterule
\title{ارجاع دادن به منابع}
\maketitle
\section{دیباچه}
احتمالاً اوّلین فردی نیستیم که در یک موضوع شروع به تحقیق می‌کنیم. بنابراین باید پژوهش‌های گذشته را مرور کرده تا «چرخ را دوباره اختراع نکنیم». حین ارائه‌ی ماحصل تحقیقِ خود، به مطالعات پیشین ارجاع می‌دهیم و نتایجِ خود را در دنباله‌ی پژوهش‌های انجام‌شده ارائه می‌کنیم \cite{shahbahrami, rankoohi}. 

منابعی که این پژوهش‌ها می‌توانند از آن‌ها به دست بیایند، در \cite{shahbahrami, rankoohi} مرور شده‌اند. صِرفِ چاپ شدنِ یک مطلب، دالِّ بر صحیح بودنِ آن نیست. به همین دلیل است که شاخص‌هایی\LTRfootnote{indexes} جهت ارزیابی منابع و ناشرها قرارداد شده‌اند تا \emph{\underline{تخمینی}} از اعتبار منابع ارائه نمایند. 

بدیهی است که وقتی در یک زمینه فعّال هستید، خودتان افراد و ناشرهای معتبر را می‌شناسید و می‌دانید که به کدام منابع می‌توان تا چه حد اعتماد کرد. در این حالت، ممکن است ببینید که شاخص‌های مطرح برای سنجشِ منابع، از جامعیّتِ کافی برخوردار نبوده و صرفاً راهنمایی برای آغازِ راهِ تحقیق هستند.
%%%%%%%%%%%%%%%%%%%%%%%%%%%%%%%%
\section{برخی از ابزارهای تسهیلِ دسترسی و ارجاع به منابع}
\subsection{\href{https://scholar.google.com}{\lr{\texttt{scholar.google.com}}}}
این وبگاه، امکانات خوبی برای جست و جوی مقالات و کتب فراهم کرده و در صورت یافتن هر یک، داده‌های مورد نیاز برای ارجاع به آن را در اختیار ما قرار می‌دهد. همچنین برخی از شاخص‌های سنجش منابع را ارائه می‌کند.

\noindent \textcolor{red}{تمرین:} اطّلاعاتی که در صفحه‌ی کاربری یک محقّق در این وبگاه دیده می‌شود را فهرست نمایید.

\noindent \textcolor{red}{تمرین:} شاخصِ «\lr{i10}» به چه معنا است؟

\noindent \textcolor{red}{تمرین:} \lr{i10}ِ اعضای گروه کامپیوتر دانشگاه گیلان چیست؟

\noindent \textcolor{red}{تمرین:} نحوه‌ی ساخت یک حساب کاربری در این وب‌گاه را بیاموزید.

\noindent در صورت تمایل، یک حساب کاربری برای خود ایجاد کنید.

\noindent \textcolor{red}{تمرین:} چه مشکلی برای ساخت یک حساب کاربری در \href{https://www.researchgate.net}{\lr{researchgate.net}} وجود دارد؟

\noindent \textcolor{red}{تمرین:} اگر فردی ۱۰۰ مقاله داشته باشد و در هر مقاله، به تمامیِ مقاله‌های قبلیِ خود ارجاع داده باشد، تعدادِ «citation»  ِ وی چند خواهد بود؟

\noindent \textcolor{red}{تمرین:} حالتی را تشریح کنید که یک مقاله، باوجود اینکه روشِ ضعیفی را ارائه کرده باشد، ارجاع‌های زیادی داشته باشد.

\noindent \textcolor{red}{تمرین:} صفحه‌ی کاربری افراد برجسته در زمینه‌ی کاری خود را در این وبگاه بیابید. اگر در هیچ زمینه‌ی دانشگاهیِ خاصّی فعّالیّت ندارید، زمینه‌ی استاد مورد علاقه‌ی خود را در نظر بگیرید. اگر به استادِ خاصّی هم علاقه ندارید، موضوعِ «ارزیابی کیفیّت تصویر\LTRfootnote{Image Quality Assessment}» را در نظر بگیرید.
%%%%%%%%%%%%%%%%%%%%%%%%%%%%%%%%%5
\subsection{\lr{\texttt{BibTeX}}}
وقتی منابع را مطالعه می‌کنیم، مطلوب است که پایگاه‌داده‌ای از آن‌ها داشته باشیم، تا به راحتی به آن‌ها ارجاع دهیم. ابزار \lr{\texttt{BibTeX}} این امکان را به شما می‌دهد که این پایگاه‌داده را در یک پرونده‌ی\LTRfootnote{file} متنی ایجاد کنید و در \LaTeX~به منابع موجود در آن ارجاع دهید.

یک مقالّه‌ی چاپ‌شده در مجلّه، با خطوط زیر در این پرونده مشخص می‌شود:
\newline
\newline
\begin{latin}
\begin{small}
\noindent \texttt{@article\{\colorbox{yellow}{krizhevsky2017imagenet},}

\noindent \texttt{title=\{Imagenet classification with deep convolutional neural networks\},}

\noindent \texttt{author=\{Krizhevsky, Alex and Sutskever, Ilya and Hinton, Geoffrey E\},}

\noindent \texttt{journal=\{Communications of the ACM\},}

\noindent \texttt{volume=\{60\},}

\noindent \texttt{number=\{6\},}

\noindent \texttt{pages=\{84--90\},}

\noindent \texttt{year=\{2017\},}

\noindent \texttt{publisher=\{AcM New York, NY, USA\}}

\noindent \texttt{\}}
\end{small}
\end{latin}
\noindent قسمت مشخّص شده، برچسب این منبع است و وقتی در \LaTeX~به آن ارجاع می‌دهیم، از این برچسب استفاده می‌کنیم: \lr{\texttt{\textbackslash cite\{krizhevsky2017imagenet\}}}. می‌توانیم این برچسب را به صورت دلخواه تغییر دهیم تا هنگام ارجاع دادن، راحت‌تر آن را به یاد بیاوریم. یک نمونه از ارجاع‌دهی به منابع فارسی و انگلیسی در پوشه‌ی \lr{\texttt{./source}} آورده شده است.

\noindent \textcolor{red}{تمرین:} با وجود کاربردهای \lr{\texttt{BibTeX}}، بسیاری از مجلّات می‌خواهند که منابع خود را با \href{https://www.overleaf.com/learn/latex/Bibliography_management_with_bibtex?&nocdn=true}{\lr{\texttt{\textbackslash bibitem}}} فهرست‌نویسی کنید. یک سند آماده کرده و در آن به یک منبع با \href{https://www.overleaf.com/learn/latex/Bibliography_management_with_bibtex?&nocdn=true}{\lr{\texttt{\textbackslash bibitem}}} ارجاع دهید.

\noindent \textcolor{red}{تمرین:} «\lr{\texttt{volume}}» و «\lr{\texttt{number}}» در فهرست‌نویسی مقالات چاپ‌شده در مجلّات به چه معنا هستند؟

\noindent \textcolor{red}{تمرین:} فرض کنید علاقه‌مند به خواندن یک مقاله هستید، ولی به آن دسترسی ندارید. یک نامه تنظیم کنید و از نویسنده‌ی مقاله بخواهید تا در صورت امکان آن را با شما به اشتراک بگذارد.

\noindent \textcolor{red}{تمرین:} تمرین قبل را برای پیاده‌سازیِ روش مطرح شده در یک مقاله تکرار کنید.
\bibliographystyle{ieeetr-fa}
\bibliography{ref}
\end{document}
