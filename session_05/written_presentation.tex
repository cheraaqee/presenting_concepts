\documentclass{article}
\usepackage{geometry}
\geometry{a4paper, top=1.5cm, bottom=1.5cm, left=2cm, right=2cm}
\usepackage{tikz}
\usetikzlibrary {arrows.meta,graphs,shapes.misc}
\usepackage{hyperref}
\usepackage{xepersian}
\settextfont{XB Niloofar}
\begin{document}
\textwidthfootnoterule
\title{ارائه‌های کتبی}
\maketitle
\section{موضوع اوّل}
در قالب یک نوشته، توضیح دهید که تجربه‌ی تحصیل کارشناسی در رشته‌ی مهندسی کامپیوتر در دانشگاه گیلان، چگونه است.

در ابتدا، باید هدف کلّی از ایجاد رشته‌ی مهندسی کامپیوتر را تشریح کنید و توضیح دهید که یک فرد در طول تحصیل در این رشته، برای چه مواردی آماده می‌شود. بدیهی است که نظر شخصیِ شما در این مورد مهم است، ولی در این قسمت باید پاسخ سوالاتِ مذکور را با استناد به منابعِ معتبر\RTLfootnote{مثلِ سرفصل مصوّبِ وزارت علوم برای مقطع کارشناسی مهندسی کامپیوتر (قابل تهیّه از وبگاه گروه: \lr{\href{https://ce.guilan.ac.ir/www/images/docs/chart92.pdf}{https://ce.guilan.ac.ir/www/images/docs/chart92.pdf}})} تهیّه کنید. همچنین باید به منابعی که برای تدریس در این رشته استفاده می‌شوند، ارجاع داده شود.

پس از معرّفیِ این رشته در حالت کلّی، توضیح دهید که این تجربه، به طور خاص، در گروه کامپیوتر دانشگاه گیلان چگونه خواهد بود. هدف این است که خواننده‌ی شما، بعد از مطالعه‌ی نوشته‌ی شما، بفهمد که:
\begin{itemize}
	\item هدف از مهندسی کامپیوتر چیست؟
	\item چرا این رشته وجود دارد؟
	\item چه منابعی در سرفصل دروس این رشته تدریس می‌شوند؟
	\item چه فرصت‌های شغلی‌ای برای من بعد از فراغت از تحصیل وجود دارد؟
	\item داشتن این موقعیّت‌های شغلی، چقدر وابسته به تحصیل در این رشته هستند؟
	\item تحصیل در این رشته، چه فرصت‌هایی فراهم می‌کند که از راه‌های دیگر قابل دستیابی نیستند؟
	\item چگونه بفهمم که به این رشته علاقه‌مند هستم یا خیر؟
	\item تجربه‌ی تحصیل این رشته در دانشگاه گیلان چگونه است؟ (امکان تخصّصی شدن در چه زمینه‌هایی آسان‌تر است؟ زندگی دانشجویی چگونه خواهد بود؟ سرفصل‌ها چه تفاوتی دارند؟ چه فرصت‌های شغلیِ خاصّی وجود دارد؟)
\end{itemize}
\subsection{کارکردِ نوشته}
\label{sec:function}
\textbf{توضیحی در موردِ «کارکرد»}:
در زندگی روی خیلی چیزها \emph{حساب باز می‌کنیم}؛ مثلاً از تاکسی انتظار داریم که مارا به ایستگاه بعدی برساند. نمی‌دانیم که آهنگ پخش می‌کند یا نه، بخاری‌اش سالم است یا قرار است که تا مقصد به دمای انجماد برسیم، ولی می‌دانیم که قرار است ما را به مقصد برساند. «کارکرد\LTRfootnote{F U N C T I O N}»ِ تاکسی، انتقالِ ما است.

موارد دیگری هستند که در زندگی برای ما کارکرد دارند. \underline{کارکردِ نوشته‌ی شما} باید این باشد که فرد سمتِ چپ در شکل~\ref{fig:function} را به فرد سمت راست تبدیل کند. فارغ از این که چقدر زیبا این کار را انجام می‌دهد، نوشته‌ی شما باید این کارکرد را داشته باشد. 
\begin{figure}
	\center
	\begin{tikzpicture}
		%%%%%%%% grid %%%%%
		% \draw[gray!20, very thin] (0, -10) grid (10, 5);
		% \foreach \x in {1,2,..., 10}
		% 	\node[gray!40] at (\x, -1pt) {\scriptsize \x};
		% \foreach \y in {-10,-9,...,5}
		% 	\node[gray!40] at (1pt, \y) {\scriptsize \y};
		%%%%%% grid %%%%%
		\draw[fill=black] (0,0) circle [radius=1.5pt];
		\draw[fill=black] (0.5, 0) circle [radius=1.5pt];
		\draw[thick] (-0.3, -0.3) -- (0.8, -0.3);
		\draw (0.3, -0.3) circle [radius=0.5cm];
		\draw (0.3, -0.8) -- (0.3, -3);
		\draw (0.3, -3) -- (0.1, -3.5);
		\draw (0.3, -3) -- (0.5, -3.5);
		\draw (0.3, -1) -- (0, -2.5);
		\draw (0.3, -1) -- (0.6, -2.8);
		%%%%%%%%%%%%
		\draw[fill=blue] (4,0) circle [radius=1.5pt];
		\draw[fill=blue] (4.5, 0) circle [radius=1.5pt];
		\draw[thick, blue] (3.7, -0.1) .. controls (4, -0.5) and (4.7, -0.5) ..(4.8, -0.1);
		\draw[blue] (4.3, -0.3) circle [radius=0.5cm];
		\draw[blue] (4.3, -0.8) -- (4.3, -3);
		\draw[blue] (4.3, -3) -- (4.1, -3.5);
		\draw[blue] (4.3, -3) -- (4.5, -3.5);
		\draw[blue] (4.3, -1) -- (4, -2.5);
		\draw[blue] (4.3, -1) -- (4.6, -2.8);
		%%%%%%%
		\draw[thick] (-1, -4) -- (1.5, -4);
		\node[align=right] at (0.2, -4.5) {\rl{قبل از خواندن نوشته‌ی شما}};
		\draw[blue, thick] (3, -4) -- (5.5, -4);
		\node[blue, align=right] at (4.2, -4.5) {\rl{بعد از خواندن نوشته‌ی شما}};
		%%%%%%
		\draw[-stealth] (1, 0) .. controls (1.5, 1) and (2.5, 1).. (3.5, 0);
		%%%%%%
		\node[rectangle, gray, draw=gray, align=right] at (-3.7, -1.5) {\rl{یک فرد عادی؛}\\\rl{- دانش‌آموزی دبیرستانی، بعد از کنکور}\\\rl{- کسی که سربازی رفته،}\\\rl{ولی تحصیلات دانشگاهی ندارد}\\\rl{- فردی که تحصیلات دانشگاهی دارد،}\\\rl{ منتها در رشته‌ای غیر از کامپیوتر}\\\rl{- فردی که تحصیلات دانشگاهی کامپیوتر دارد،}\\\rl{منتها در دانشگاهی غیر از دانشگاه گیلان}};
		\node[rectangle, blue, draw=blue, align=right] at (2, -9) {
			\rl{می‌دانم که:}
		\\\rl{- هدف از مهندسی کامپیوتر چیست؟}
		\\\rl{- چرا این رشته وجود دارد؟}
		\\\rl{- چه منابعی در سرفصل دروس این رشته تدریس می‌شوند؟}
		\\\rl{- چه فرصت‌های شغلی‌ای برای من بعد}\\\rl{از فراغت از تحصیل وجود دارد؟}
		\\\rl{- داشتن این موقعیّت‌های شغلی، چقدر وابسته به}\\\rl{تحصیل در این رشته هستند؟}
		\\\rl{- تحصیل در این رشته، چه فرصت‌هایی فراهم}\\\rl{می‌کند که از راه‌های دیگر قابل دستیابی نیستند؟}
		\\\rl{- چگونه بفهمم که به این رشته علاقه‌مند هستم یا خیر؟}
		\\\rl{- تجربه‌ی تحصیل این رشته}\\\rl{در دانشگاه گیلان چگونه است؟}\\\rl{(امکان تخصّصی شدن در چه زمینه‌هایی آسان‌تر است؟}\\\rl{زندگی دانشجویی چگونه خواهد بود؟}\\\rl{سرفصل‌ها چه تفاوتی دارند؟}\\\rl{چه فرصت‌های شغلیِ خاصّی وجود دارد؟)}
		\\\rl{- نظر نویسنده در این‌باره چیست و چه پیشنهادها و انتقادهایی دارد.}
};
	\end{tikzpicture}
	\caption{کارکردِ نوشته‌ی شما درباره‌ی مهندسی کامپیوتر در دانشگاه گیلان}
	\label{fig:function}
\end{figure}

بعد از طرح موارد فوق، نظر خود را اضافه کرده و هرگونه انتقاد و پیشنهاد به موارد مطرح شده را تشریح نمایید. در نهایت یک قسمت جمع‌بندی برای نوشته‌ی خود نگارش کنید. یک ساختار پیشنهادی برای نوشته در \lr{\href{./uni.pdf}{\texttt{uni.pdf}}} ارائه شده است\footnote{اجباری نیست که ترتیب ارائه‌ی مطالبِ شما، مطابق این ساختار پیشنهادی باشد.}.
\section{موضوع دوم}
اگر در یک زمینه‌ی تخصصی فعّال هستید، یک پیشنهاده\footnote{معادلِ فارسی برای «\lr{proposal}»} تنظیم کنید و مهارت خود را در قالب یک طرح عرضه کنید. 

یک مشتری یا صاحبِ سرمایه، به مشکل یا نیازی دچار است و شما قادرید این نیاز را با هزینه و زمانِ مشخّص مرتفع سازید. در پیشنهاده‌ی خود این نیاز را تشریح کرده و کارهای انجام‌شده در آن موضوع را مرور و تحلیل می‌کنید. کارهای انجام‌شده هم محصول‌ها و طرح‌های صنعتی هستند، هم پژوهش‌ها و مقاله‌های دانشگاهی\footnote{بدیهی است که دنیای «open-source» از این مرور مستثنی نیست.}.

سپس توضیح می‌دهید که شما چگونه آن مسئله را حل خواهید کرد. منابع و هزینه‌هایی که نیاز دارید نیز باید تشریح شوند. مراحل انجام طرح‌تان را فهرست کرده و یک نمودارِ گانت\LTRfootnote{Gantt chart} از زمان اجرای طرح‌تان ارائه کنید.
\subsection{کارکردِ پیشنهاده}
پیشنهاده‌ی شما باید کارکردهای\footnote{برای توضیحِ «کارکرد»، رجوع شود به~\ref{sec:function}} زیر را داشته باشد:
\begin{itemize}
	\item مشتری، کارفرما، یا صاحب‌سرمایه، بفهمد که مسئله چیست؟
	\item مشتری درک کند که وضعیّت فناوری درباره‌ی حلِّ این مسئله چگونه است؟ (چه راه حلّی، با چه هزینه‌ای در دسترس است؟)
	\item مشتری و کارشناسان فنّیِ معتمد وِی، قانع شوند که شما می‌توانید برایشان سودمند باشید.
	\item کارشناسان فنّیِ معتمدِ مشتری، درک کنند که شما چگونه می‌خواهید مسئله را حل کنید\footnote{دقّت داشته باشید که در حالِ حلِّ یک مسئله‌ی بنیادی نیستید که پاسخی انقلابی داشته و حتّی جنبه‌های نظریِ آن بر کارشناسان پوشیده باشد. عمده‌ی برتری شما، باید در نحوه‌ی پیاده‌سازیِ راهِ حل‌های معلوم جلوه داده شود.}.
	\item مشتری سطحِ قابلِ قبولی از خاطرجمعی در مورد تحویلِ محصول در زمانِ تعیین‌شده‌ی شما داشته باشد.
\end{itemize}
یک قالب برای پیشنهاده در \href{./proposal.pdf}{\texttt{proposal.pdf}} موجود است.
\end{document}
